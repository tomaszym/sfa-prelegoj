\documentclass{beamer}
\usepackage[T1]{fontenc}
\usepackage[utf8x]{inputenc} 
\usepackage[esperanto,polish]{babel}
%\usepackage{umbcboxes}

\title[Esperanto w programowaniu]{Zalety używania planowego języka aglutacyjnego do programowania na przykładzie esperanta.}
%\subtitle{}
\author{Tomasz Szymula}
\institute[PEJ]{Koło naukowe Blabel, AGH}

\date[2014]{Nta sesja kół naukowych, 2014}
\subject{Lingvistiko}
%Zalety używania planowego języka aglutacyjnego do programowania na przykładzie esperanta.

%Język angielski jest standardem, jeśli chodzi o język w którym pisany
%jest zarówno kod (np identyfikatory) jak i komentarze. Zastosowanie do
%tego celu języka z regularnym słowotwórstwem opartym o sklejanie
%zrostków daje interesujące rezultaty. Omówienie kilku przydatnych
%afiksów esperanta z przykładami oraz porównaniem do języka angielskiego
\usetheme{Warsaw}
%\usecolortheme{crane}

\AtBeginSection[]
{
  \begin{frame}
    \frametitle{Plano}
    \tableofcontents[currentsection]
  \end{frame}
}

\begin{document}
  \frame{\titlepage}
 
  \begin{frame}
  	\frametitle{Codzienność}
  	
	\begin{itemize}
		\item Używam esperanta w kontaktach z przyjaciółmi
		\item Programuję w Scali w pracy i w projektach na uczelnię
	\end{itemize}
  
  \end{frame}
 

  \section{Esperanto}
  \subsection{Czym jest} 
  
  \begin{frame}
  	\frametitle{Esperanto}
  	
  	\begin{block}{Esperanto to język}
  		\begin{itemize}
  			\item planowy,
			\item międzynarodowy,
			\item pomocniczy.
		\end{itemize}
  	\end{block}
  	
  	\pause

	\begin{block}{A dla lingwisty}
		Aglutynacyjny. 
	\end{block}  	
  	
  \end{frame}
 
  
  
  \subsection{Jakie jest}
  
  \begin{frame}
  	\frametitle{Rdzenie}
	
	O tym skąd są rdzenie

  \end{frame}
  
  \begin{frame}
  	\frametitle{Afiksy}

	Na trzech slajdach będą chyba trzy najważniejsze z perspektywy późniejszych przykładów sufiksy:\\
		-il: narzędzie \\ kompili - kompilować $\rightarrow$ kompililo - kompilator \\
		-ar: zbiór\\ klavo - przycisk $\rightarrow$ klavaro - klawiatura\\
		-er: cząstka \\ programo - program $\rightarrow$ programero - część programu\\

  \end{frame}

  \begin{frame}
  	\frametitle{Ogólny przykład do afiksów}
  	
  	coś w stylu:\\
  	manĝi - jesc,\\
  	manĝilo - sztuciec\\
  	manĝilaro - zastawa\\
  	manĝeti - podjadac\\
  	manĝegi - "wcinac"\\
  	antaŭmanĝi - przedjadac (?)\\
  	itd.

  \end{frame}
 
  
  \begin{frame}
    \frametitle{Ekzemplo: Ludo de vivo}
  \end{frame}
  
  \section{W praktyce}
  \subsection{Banda czworga}
  
  \begin{frame}
  \frametitle{Wzorce projektowe}
  Wzorce projektowe to dobry przykład, żeby najpierw pokazać, że w ogólnych przypadkach esperanto radziłoby sobie niegorzej niż angielski (który chwilami też jest aglutynacyjny z resztą).
  \end{frame}  

  \begin{frame}
  \frametitle{Obserwator, Obserwowany}
  Spektato, Spektanto
  \end{frame}  
  
  \begin{frame}
  \frametitle{Dekorator}
  Dekorilo
  \end{frame}  
  
  \subsection{Moje przykłady}
  \begin{frame}
  	\frametitle{Kolekcja użytkowników}

	uzantaro vs. users, userList, userCollection
	
	uzantaroj?
  \end{frame}
  
  \begin{frame}
  	\frametitle{Część}

	imagePart vs bildero,\\
	recivatero
  \end{frame}
  
  \begin{frame}
    \frametitle{Inne przykłady}
	
	Jeszcze jakieś przykłady oczywiście bym znalazł - dopasuję w zależności od tego ile mi zajmie powiedzenie tego co już jest.
  \end{frame}  
  
  \begin{frame}
  	\frametitle{Podsumowanie}
  	Jeśli 
  	\begin{itemize}
  		\item opracowanoby nowy, lepszy niż esperanto język sztuczny do używania przez programistów,
  		\item zostałby natychmiast wprowadzony do nauki w uniwersytetach oraz uzyskał wsparcie biznesu
  	\end{itemize}
  	
  	z perspektywy ludzkości -- czy ,,inwestycja'' zwróciłaby~się? Po jakim czasie?
  \end{frame}

\end{document}